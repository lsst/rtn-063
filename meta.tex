% GENERATED FILE -- edit this in the Makefile
\newcommand{\lsstDocType}{RTN}
\newcommand{\lsstDocNum}{063}
\newcommand{\vcsRevision}{8d1f7ab-dirty}
\newcommand{\vcsDate}{2023-08-16}
