\documentclass[OPS,authoryear,toc]{lsstdoc}
% GENERATED FILE -- edit this in the Makefile
\newcommand{\lsstDocType}{RTN}
\newcommand{\lsstDocNum}{063}
\newcommand{\vcsRevision}{ec4d71a-dirty}
\newcommand{\vcsDate}{2023-08-18}


% Package imports go here.

% Local commands go here.

%If you want glossaries
%\input{aglossary.tex}
%\makeglossaries

\title{HSC PDR2 Reprocessing and Operations Rehearsal for DRP}

% Optional subtitle
% \setDocSubtitle{A subtitle}

\author{%
Yusra AlSayyad
}

\setDocRef{RTN-063}
\setDocUpstreamLocation{\url{https://github.com/lsst/rtn-063}}

\date{\vcsDate}

% Optional: name of the document's curator
% \setDocCurator{The Curator of this Document}

\setDocAbstract{%
In 2023, the campaign management team processed 100s of sq degrees of precursor data though a data release production on the new US data facility at SLAC.  This activity demonstrated the production of a data release under simulated operational conditions. 
}

% Change history defined here.
% Order: oldest first.
% Fields: VERSION, DATE, DESCRIPTION, OWNER NAME.
% See LPM-51 for version number policy.
\setDocChangeRecord{%
  \addtohist{1}{YYYY-MM-DD}{Unreleased.}{Yusra AlSayyad}
}


\begin{document}

% Create the title page.
\maketitle
% Frequently for a technote we do not want a title page  uncomment this to remove the title page and changelog.
% use \mkshorttitle to remove the extra pages

% ADD CONTENT HERE
% You can also use the \input command to include several content files.

\appendix
% Include all the relevant bib files.
% https://lsst-texmf.lsst.io/lsstdoc.html#bibliographies
\section{References} \label{sec:bib}
\renewcommand{\refname}{} % Suppress default Bibliography section
\bibliography{local,lsst,lsst-dm,refs_ads,refs,books}

% Make sure lsst-texmf/bin/generateAcronyms.py is in your path
\section{Acronyms} \label{sec:acronyms}
\addtocounter{table}{-1}
\begin{longtable}{p{0.145\textwidth}p{0.8\textwidth}}\hline
\textbf{Acronym} & \textbf{Description}  \\\hline

2MASS & Two-Micron All Sky Survey \\\hline
BPS & Batch Production Service \\\hline
CCD & Charge-Coupled Device \\\hline
CERN & European Organization for Nuclear Research \\\hline
CM & Configuration Management \\\hline
DCR & Differential Chromatic Refraction \\\hline
DEC & Declination \\\hline
DEEP & Deep Extragalactic Evolutionary Probe \\\hline
DM & Data Management \\\hline
DMTN & DM Technical Note \\\hline
DP1 & Data Preview 1 \\\hline
DP2 & Data Preview 2 \\\hline
DR1 & Data Release 1 \\\hline
DRP & Data Release Production \\\hline
DS9 & Deep Space 9 (specific astronomical data visualisation application; SAOImage) \\\hline
EUPS & Extended Unix Product System \\\hline
FGCM & Forward Global Calibration Model \\\hline
GB & Gigabyte \\\hline
HSC & Hyper Suprime-Cam \\\hline
ISR & Instrument Signal Removal \\\hline
LSST & Legacy Survey of Space and Time (formerly Large Synoptic Survey Telescope) \\\hline
MB & MegaByte \\\hline
NAOJ & National Astronomical Observatory of Japan \\\hline
OPS & Operations \\\hline
PDR & Preliminary Design Review \\\hline
PDR2 & Public Data Release 2 (HSC) \\\hline
POSIX & Portable Operating System Interface \\\hline
PSF & Point Spread Function \\\hline
PanDA &  Production ANd Distributed Analysis system \\\hline
RA & Right Ascension \\\hline
RAM & Random Access Memory \\\hline
RTN & Rubin Technical Note \\\hline
SLAC & SLAC National Accelerator Laboratory \\\hline
UK & United Kingdom \\\hline
US & United States \\\hline
USDF & United States Data Facility \\\hline
VM & Virtual Machine \\\hline
YAML & Yet Another Markup Language \\\hline
bps & bit(s) per second \\\hline
\end{longtable}

% If you want glossary uncomment below -- comment out the two lines above
%\printglossaries





\end{document}
